\documentclass[12pt, titlepage]{article}

\usepackage{amsmath, mathtools}
\usepackage{siunitx}
\usepackage{fullpage}
% \usepackage[round]{natbib}
\usepackage[numbers,square]{natbib}
\usepackage{multirow}
\usepackage{booktabs}
\usepackage{tabularx}
\usepackage{graphicx}
\usepackage{float}
\usepackage{hyperref}
\hypersetup{
    colorlinks,
    citecolor=blue,
    filecolor=black,
    linkcolor=red,
    urlcolor=blue
}

\input{../../Comments}
%% Common Parts

\newcommand{\progname}{ImgBeamer} % PUT YOUR PROGRAM NAME HERE
\newcommand{\prognamelong}{SEM image formation demo tool}
\newcommand{\authname}{Joachim de Fourestier} % AUTHOR NAMES                  

% for code fragments
\newcommand{\code}[1]{\texttt{#1}}

% Useful for tables - top-left centered columns
% https://tex.stackexchange.com/a/337681/289389
\newcolumntype{L}[1]{>{\raggedright\arraybackslash}p{#1}}

\usepackage{hyperref}
    \hypersetup{colorlinks=true, linkcolor=blue, citecolor=blue, filecolor=blue,
                urlcolor=blue, unicode=false}
    \urlstyle{same}
                                


\newcounter{acnum}
\newcommand{\actheacnum}{AC\theacnum}
\newcommand{\acref}[1]{AC\ref{#1}}

\newcounter{ucnum}
\newcommand{\uctheucnum}{UC\theucnum}
\newcommand{\uref}[1]{UC\ref{#1}}

\newcounter{mnum}
\newcommand{\mthemnum}{M\themnum}
\newcommand{\mref}[1]{M\ref{#1}}

\begin{document}

\title{Module Guide for \progname{}} 
\author{\authname}
\date{\today}

\maketitle

\pagenumbering{roman}

\section{Revision History}

\begin{tabularx}{\textwidth}{p{3cm}p{2cm}X}
\toprule {\bf Date} & {\bf Version} & {\bf Notes}\\
\midrule
2023/03/05 & 0.1.0 & Creation\\
2023/03/10 & 0.1.1 & Add anticipated changes\\
2023/03/12 & 0.1.2 & Add unlikely changes\\
           & 0.1.3 & Add the module hierarchy and Behaviour-Hiding module descriptions\\
\bottomrule
\end{tabularx}

\newpage

\section{Reference Material}

This section records information for easy reference.

\subsection{Abbreviations and Acronyms}

\renewcommand{\arraystretch}{1.2}
\begin{tabular}{l l} 
  \toprule		
  \textbf{symbol} & \textbf{description}\\
  \midrule 
  AC & Anticipated Change\\
  DAG & Directed Acyclic Graph \\
  DOM & Document Object Model \\
  M & Module \\
  MG & Module Guide \\
  OS & Operating System \\
  R & Requirement\\
  RBGA & Red-Blue-Green-Alpha (pixel components)\\
  SC & Scientific Computing \\
  SRS & Software Requirements Specification\\
  \progname & SEM image formation demo tool\\
  UC & Unlikely Change \\
  \bottomrule
\end{tabular}\\

\newpage

\tableofcontents

\listoftables

\listoffigures

\newpage

\pagenumbering{arabic}

\section{Introduction}

Decomposing a system into modules is a commonly accepted approach to developing
software.  A module is a work assignment for a programmer or programming
team~\citep{ParnasEtAl1984}.  We advocate a decomposition
based on the principle of information hiding~\citep{Parnas1972a}.  This
principle supports design for change, because the ``secrets'' that each module
hides represent likely future changes.  Design for change is valuable in SC,
where modifications are frequent, especially during initial development as the
solution space is explored. \newline

Our design follows the rules layed out by \citet{ParnasEtAl1984}, as follows:
\begin{itemize}
\item System details that are likely to change independently should be the
  secrets of separate modules.
\item Each data structure is implemented in only one module.
\item Any other program that requires information stored in a module's data
  structures must obtain it by calling access programs belonging to that module.
\end{itemize}

After completing the first stage of the design, the Software Requirements
Specification (SRS), the Module Guide (MG) is developed~\citep{ParnasEtAl1984}. The MG
specifies the modular structure of the system and is intended to allow both
designers and maintainers to easily identify the parts of the software.  The
potential readers of this document are as follows:

\begin{itemize}
\item New project members: This document can be a guide for a new project member
  to easily understand the overall structure and quickly find the
  relevant modules they are searching for.
\item Maintainers: The hierarchical structure of the module guide improves the
  maintainers' understanding when they need to make changes to the system. It is
  important for a maintainer to update the relevant sections of the document
  after changes have been made.
\item Designers: Once the module guide has been written, it can be used to
  check for consistency, feasibility, and flexibility. Designers can verify the
  system in various ways, such as consistency among modules, feasibility of the
  decomposition, and flexibility of the design.
\end{itemize}

The rest of the document is organized as follows. Section
\ref{SecChange} lists the anticipated and unlikely changes of the software
requirements. Section \ref{SecMH} summarizes the module decomposition that
was constructed according to the likely changes. Section \ref{SecConnection}
specifies the connections between the software requirements and the
modules. Section \ref{SecMD} gives a detailed description of the
modules. Section \ref{SecTM} includes two traceability matrices. One checks
the completeness of the design against the requirements provided in the SRS. The
other shows the relation between anticipated changes and the modules. Section
\ref{SecUse} describes the use relation between modules.

\section{Anticipated and Unlikely Changes} \label{SecChange}

This section lists possible changes to the system. According to the likeliness
of the change, the possible changes are classified into two
categories. Anticipated changes are listed in Section \ref{SecAchange}, and
unlikely changes are listed in Section \ref{SecUchange}.

\subsection{Anticipated Changes} \label{SecAchange}

Anticipated changes are the source of the information that is to be hidden
inside the modules. Ideally, changing one of the anticipated changes will only
require changing the one module that hides the associated decision. The approach
adapted here is called design for
change.

\begin{description}
  \item[\refstepcounter{acnum} \actheacnum \label{AC_hardware}:] The specific
    hardware on which the software is running.
  \item[\refstepcounter{acnum} \actheacnum \label{AC_input}:] The format of the
    initial input data.
  \item[\refstepcounter{acnum} \actheacnum \label{AC_inFormat}:] The format of the input parameters.
  \item[\refstepcounter{acnum} \actheacnum \label{AC_inVerif}:] The constraints on the input parameters.
  \item[\refstepcounter{acnum} \actheacnum \label{AC_outFormat}:] The format of the final output data.
  \item[\refstepcounter{acnum} \actheacnum \label{AC_outVerif}:] The constraints on the output results.
  \item[\refstepcounter{acnum} \actheacnum \label{AC_calc}:] How the overall control of the
    calculations are orchestrated.
  \item[\refstepcounter{acnum} \actheacnum \label{AC_displayUnits}:] The display of scale in
    virtual, physical, or real world units such as \si{nm}.
  \item[\refstepcounter{acnum} \actheacnum \label{AC_simNoise}:] The ability to simulate basic
    image noise (such as Gaussian or Poisson).
  \item[\refstepcounter{acnum} \actheacnum \label{AC_imgMetricAlgos}:] The ability
    to use different image quality metrics.
  \item[\refstepcounter{acnum} \actheacnum \label{AC_maskOpts}:] The implementation of compositing
    or blending operations (e.g., ``Bit BLT'').
  \item[\refstepcounter{acnum} \actheacnum \label{AC_calcSignal}:] The calculation of the
    average pixel or signal (intensity) value sampled by the beam.
  \item[\refstepcounter{acnum} \actheacnum \label{AC_GUI}:] The implementation of graphical user
    interface.
\end{description}

\subsection{Unlikely Changes} \label{SecUchange}

The module design should be as general as possible. However, a general system is
more complex. Sometimes this complexity is not necessary. Fixing some design
decisions at the system architecture stage can simplify the software design. If
these decision should later need to be changed, then many parts of the design
will potentially need to be modified. Hence, it is not intended that these
decisions will be changed.

\begin{description}
  \item[\refstepcounter{ucnum} \uctheucnum \label{UC_IO}:] Input/Output devices
    (Input: File and/or Keyboard, Output: File, Memory, and/or Screen).
  \item[\refstepcounter{ucnum} \uctheucnum \label{UC_physics}:] The software will not
    simulate electron beam physics, collision cascades, sample nature, topography, 
    or other electron-sample interactions.
  \item[\refstepcounter{ucnum} \uctheucnum \label{UC_layout}:] The beam sampling
    layout or raster pattern.
  \item[\refstepcounter{ucnum} \uctheucnum \label{UC_import}:] The software must
    be able to import a groud truth image.
  \item[\refstepcounter{ucnum} \uctheucnum \label{UC_export}:] The software must
    be able to export the resulting image.
  \item[\refstepcounter{ucnum} \uctheucnum \label{UC_visualize}:] The software must be able
    to display a live visualization to the user of the effects of changing the imaging parameters.
\end{description}

\section{Module Hierarchy} \label{SecMH}

This section provides an overview of the module design. Modules are summarized
in a hierarchy decomposed by secrets in Table \ref{TblMH}. The modules listed
below, which are leaves in the hierarchy tree, are the modules that will
actually be implemented.

\begin{description}
\item [\refstepcounter{mnum} \mthemnum \label{M_HdwHide}:] Hardware-Hiding Module
\item [\refstepcounter{mnum} \mthemnum \label{M_imgGTInput}:] Ground Truth Image Input
\item [\refstepcounter{mnum} \mthemnum \label{M_params}:] Imaging Parameters Input
\item [\refstepcounter{mnum} \mthemnum \label{M_vizSpotProfile}:] Spot Profile Control
\item [\refstepcounter{mnum} \mthemnum \label{M_export}:] Image Export
\item [\refstepcounter{mnum} \mthemnum \label{M_infoDisp}:] Information and Metrics Display
\item [\refstepcounter{mnum} \mthemnum \label{M_metric}:] Image Metrics Calculation
\item [\refstepcounter{mnum} \mthemnum \label{M_drawStage}:] Drawing Stage / Canvas Module
\item [\refstepcounter{mnum} \mthemnum \label{M_compositing}:] Image Compositing
\item [\refstepcounter{mnum} \mthemnum \label{M_rendering}:] Image Compositing
\item [\refstepcounter{mnum} \mthemnum \label{M_sampling}:] Image Sampling
\item [\refstepcounter{mnum} \mthemnum \label{M_DOM}:] DOM Manipulation
\item [\refstepcounter{mnum} \mthemnum \label{M_vizGT}:] Ground Truth Visualization
\item [\refstepcounter{mnum} \mthemnum \label{M_vizSubregion}:] Subregion Visualization
\item [\refstepcounter{mnum} \mthemnum \label{M_vizSpotContent}:] Spot Content Visualization
\item [\refstepcounter{mnum} \mthemnum \label{M_vizSpotSignal}:] Spot Signal Visualization
\item [\refstepcounter{mnum} \mthemnum \label{M_vizSpotLayout}:] Spot Layout Visualization
\item [\refstepcounter{mnum} \mthemnum \label{M_vizSampledSub}:] Sampled Subregion Visualization
\item [\refstepcounter{mnum} \mthemnum \label{M_vizResultSub}:] Resulting Subregion Visualization
\item [\refstepcounter{mnum} \mthemnum \label{M_vizResultImg}:] Resulting Image Visualization
\end{description}


\begin{table}[h!]
\centering
\begin{tabular}{p{0.3\textwidth} p{0.3\textwidth} p{0.4\textwidth}}
\toprule
\textbf{Level 1} & \textbf{Level 2} & \textbf{Level 3}\\
\midrule

{Hardware-Hiding Module} & ~ \\
\midrule

\multirow{13}{0.3\textwidth}{Behaviour-Hiding Module}
& \multirow{3}{0.3\textwidth}{Input}
  & Ground Truth Image Input \\
  && Imaging Parameters Input \\
  && Spot Profile Control \\
  \cline{2-3}
& \multirow{2}{0.3\textwidth}{Output}
  & Information and Metrics Display \\
  && Image Export\\
  \cline{2-3}
& \multirow{8}{0.3\textwidth}{Visualization}
  & Ground Truth \\
  && Subregion \\
  && Spot Content \\
  && Spot Signal \\
  && Spot Layout \\
  && Sampled Subregion \\
  && Resulting Subregion \\
  && Resulting Image \\
\midrule

\multirow{6}{0.3\textwidth}{Software Decision Module}
& \multirow{2}{0.3\textwidth}{Graphical Display}
  & Drawing Stage / Canvas Module \\
  && DOM Manipulation \\
  \cline{2-3}
& \multirow{4}{0.3\textwidth}{Image Manipulation and Processing}
  & Image Data Structure \\
  && Image Compositing \\
  && Image Rendering \\
  && Image Sampling \\
  && Image Metrics Calculation \\
\bottomrule

\end{tabular}
\caption{Module Hierarchy}
\label{TblMH}
\end{table}

\section{Connection Between Requirements and Design} \label{SecConnection}

The design of the system is intended to satisfy the requirements developed in
the SRS. In this stage, the system is decomposed into modules. The connection
between requirements and modules is listed in Table~\ref{TblRT}.

\section{Module Decomposition} \label{SecMD}

Modules are decomposed according to the principle of ``information hiding''
proposed by \citet{ParnasEtAl1984}. The \emph{Secrets} field in a module
decomposition is a brief statement of the design decision hidden by the
module. The \emph{Services} field specifies \emph{what} the module will do
without documenting \emph{how} to do it. For each module, a suggestion for the
implementing software is given under the \emph{Implemented By} title. If the
entry is \emph{OS}, this means that the module is provided by the operating
system or by standard programming language libraries.
If the entry is \emph{Web browser}, this means that the module is provided by the 
HTML 5 compliant web browser.
If the entry is \emph{CanvasAPI}, this means that the module is provided by the 
Canvas API of the HTML 5 living standard \cite{html_std_canvas}.
If the entry is \emph{Konva}, this means that the module is provided by the 
Konva.js HTML5 2d canvas javascript library \cite{konva_2021}.
\emph{\progname{}} means the module will be implemented by the \progname{} software.

~\newline
Only the leaf modules in the hierarchy have to be implemented. If a dash
(\emph{--}) is shown, this means that the module is not a leaf and will not have
to be implemented.

\subsection{Hardware Hiding Modules (\mref{M_HdwHide})}

\begin{description}
\item[Secrets:]The data structure and algorithm used to implement the virtual
  hardware.
\item[Services:]Serves as a virtual hardware used by the rest of the
  system. This module provides the interface between the hardware and the
  software. So, the system can use it to display outputs or to accept inputs.
\item[Implemented By:] OS
\end{description}

\subsection{Behaviour-Hiding Module}

\begin{description}
\item[Secrets:]The contents of the required behaviours.
\item[Services:]Includes programs that provide externally visible behaviour of
  the system as specified in the software requirements specification (SRS)
  documents. This module serves as a communication layer between the
  hardware-hiding module and the software decision module. The programs in this
  module will need to change if there are changes in the SRS.
\item[Implemented By:] --
\end{description}


\subsubsection{Ground Truth Image Input (\mref{M_imgGTInput})}
\begin{description}
\item[Secrets:]The format, structure, verification of the input ground truth image data.
\item[Services:]Converts the input ground truth image data into the data structure used by the
  visualization modules \mref{M_vizGT}, \mref{M_vizSubregion}, and \mref{M_vizResultImg}.
\item[Implemented By:] CanvasAPI
\item[Type of Module:] \an{[Record, Library, Abstract Object, or Abstract Data Type]
  [Information to include for leaf modules in the decomposition by secrets tree.]}
\end{description}


\subsubsection{Imaging Parameters Input (\mref{M_params})}
\begin{description}
\item[Secrets:]The format, structure, GUI of the input imaging parameters (number
  of rows and columns that will be used for the sampling/imaging raster grid) data.
\item[Services:]Displays and converts the input data into the data structure used by the
  visualization modules \mref{M_vizSpotLayout}, \mref{M_vizSampledSub}, \mref{M_vizResultSub},
  and \mref{M_vizResultImg}.
\item[Implemented By:] \progname{}
\item[Type of Module:] \an{[Record, Library, Abstract Object, or Abstract Data Type]
  [Information to include for leaf modules in the decomposition by secrets tree.]}
\end{description}


\subsubsection{Spot Profile Control (\mref{M_vizSpotProfile})}
\begin{description}
\item[Secrets:]The format and structure of the input data (spot shape) provided
  through by user to GUI interaction.
\item[Services:] Obtains the shape of the spot as provided by the user and stores it for use
  by the visualization modules \mref{M_vizSpotContent}, \mref{M_vizSpotSignal}, \mref{M_vizSpotLayout},
  \mref{M_vizSampledSub}, \mref{M_vizResultSub}, and \mref{M_vizResultImg}.
\item[Implemented By:] \progname{}
\item[Type of Module:] \an{[Record, Library, Abstract Object, or Abstract Data Type]
  [Information to include for leaf modules in the decomposition by secrets tree.]}
\end{description}


\subsubsection{Image Export (\mref{M_export})}
\begin{description}
\item[Secrets:]The format and structure of the data of the resulting image.
\item[Services:]Converts the resulting image, as provided by the 
  Resulting Image Visualization module (\mref{M_vizResultImg}), to an output image file.
\item[Implemented By:] CanvasAPI, Konva, and ImgBeamer
\item[Type of Module:] \an{[Record, Library, Abstract Object, or Abstract Data Type]
  [Information to include for leaf modules in the decomposition by secrets tree.]}
\end{description}


\subsubsection{Information and Metrics Display (\mref{M_infoDisp})}
\begin{description}
\item[Secrets:]The display and calculation methods of the information on rendered
  images from the visualization modules (\mref{M_vizSubregion}, \mref{M_vizSpotProfile},
  \mref{M_vizSpotContent}, \mref{M_vizSpotSignal}, and \mref{M_vizResultImg}),
  the input ground truth image (\mref{M_imgGTInput}), and the spot profile/shape 
  information (width, height, rotation, and eccentricity).
\item[Services:]Calculates and displays information such as the magnification,
  image quality metrics, spot shape information, spot signal pixel values (RBGA),
  and the drawing rate of resulting image (e.g., rows per milliseconds).
\item[Implemented By:] \progname{}
\item[Type of Module:] \an{[Record, Library, Abstract Object, or Abstract Data Type]
  [Information to include for leaf modules in the decomposition by secrets tree.]}
\end{description}


\subsubsection{Ground Truth Visualization (\mref{M_vizGT})}
\begin{description}
\item[Secrets:]The display (sizing, style, colours, etc.) to represent the ground truth image
  the display of the subregion bounds, and drawing stage.
\item[Services:]Converts the image data, as provided by Ground Truth Image Input (\mref{M_imgGTInput}),
  and displays it to the user. And, draws/display a highlighted area representing the subregion
  represented in the Subregion Visualization (\mref{M_vizSubregion}).
\item[Implemented By:] \progname{}
\item[Type of Module:] \an{[Record, Library, Abstract Object, or Abstract Data Type]
  [Information to include for leaf modules in the decomposition by secrets tree.]}
\end{description}


\subsubsection{Subregion Visualization (\mref{M_vizSubregion})}
\begin{description}
\item[Secrets:]The bounds (size and location) of the subregion within the ground truth image,
  the display (scaling, colours, styling), and drawing stage.
\item[Services:]Displays the image representing the subregion and obtains GUI
  input (pan and zoom) from the user.
\item[Implemented By:] \progname{}
\item[Type of Module:] \an{[Record, Library, Abstract Object, or Abstract Data Type]
  [Information to include for leaf modules in the decomposition by secrets tree.]}
\end{description}


\subsubsection{Spot Content Visualization (\mref{M_vizSpotContent})}
\begin{description}
\item[Secrets:]A separate subregion representation that is scaled (zoomed) according to
  the spot size, a display for the ``stencil'' representation based on the spot profile,
  and drawing stage.
\item[Services:]Serves as example to how the image is sampled (or ``stenciled'').
  Takes a copy of the subregion (from \mref{M_vizSubregion}) image and the spot shape, 
  as provided the Spot Profile Control (\mref{M_vizSpotProfile}), to display a ``stenciled'' image
  that represents the image content that will be use produce a signal value.
  (as displayed by \mref{M_vizSpotSignal}).
\item[Implemented By:] \progname{}
\item[Type of Module:] \an{[Record, Library, Abstract Object, or Abstract Data Type]
  [Information to include for leaf modules in the decomposition by secrets tree.]}
\end{description}


\subsubsection{Spot Signal Visualization (\mref{M_vizSpotSignal})}
\begin{description}
\item[Secrets:]The calculation or determination of the signal (pixel) value to represent
  what has been sampled by the spot or beam, the display (scaling, colours, styling),
  and drawing stage.
\item[Services:]Calculates a signal value (pixel colour) to use as the fill colour
  of a copy of the spot profile to display to the user.
  Serves as example of what the sampled spot looks like based on the image content
  (as provided and displayed by \mref{M_vizSpotContent}).
\item[Implemented By:] \progname{}
\item[Type of Module:] \an{[Record, Library, Abstract Object, or Abstract Data Type]
  [Information to include for leaf modules in the decomposition by secrets tree.]}
\end{description}


\subsubsection{Spot Layout Visualization (\mref{M_vizSpotLayout})}
\begin{description}
\item[Secrets:]The display (scaling, colours, styling) and drawing stage of the spot layout.
\item[Services:]Given the subregion image (from \mref{M_vizSubregion}), 
  the spot size (provided by \mref{M_vizSpotContent}),
  and the spot shape (provided by \mref{M_vizSpotProfile}),
  draws and display the sampling/imaging raster grid
  (number of rows and columns are provided by \mref{M_params}) and
  the spots located in each cell. This displays how and what areas of the subregion
  image that are sampled and will be used to visualize the 
  sampled region (by \mref{M_vizSampledSub}).
\item[Implemented By:] \progname{}
\item[Type of Module:] \an{[Record, Library, Abstract Object, or Abstract Data Type]
  [Information to include for leaf modules in the decomposition by secrets tree.]}
\end{description}


\subsubsection{Sampled Subregion Visualization (\mref{M_vizSampledSub})}
\begin{description}
\item[Secrets:]The display (scaling, colours, styling) and drawing stage 
  of the subregion sampled image content.
\item[Services:]Given the spot size (\mref{M_vizSpotContent}) and
  shape (\mref{M_vizSpotProfile}), and the sampling raster grid (\mref{M_params}),
  the module should display the sampled (``stenciled'') image content from the areas
  depicted by module \mref{M_vizSpotLayout}.
\item[Implemented By:] \progname{}
\item[Type of Module:] \an{[Record, Library, Abstract Object, or Abstract Data Type]
  [Information to include for leaf modules in the decomposition by secrets tree.]}
\end{description}


\subsubsection{Resulting Subregion Visualization (\mref{M_vizResultSub})}
\begin{description}
\item[Secrets:]The display (scaling, colours, styling) and drawing stage 
  of the resulting subregion.
\item[Services:]Displays the resulting subregion image based on the sampling
  (as depicted by \mref{M_vizSpotLayout}),
  where each cell (or pixel) in the raster grid (as defined by \mref{M_params})
  is filled with the calculated sampled signal value (or pixel colour) based
  on the sampled image content (as depicted by \mref{M_vizSampledSub}).
\item[Implemented By:] \progname{}
\item[Type of Module:] \an{[Record, Library, Abstract Object, or Abstract Data Type]
  [Information to include for leaf modules in the decomposition by secrets tree.]}
\end{description}


\subsubsection{Resulting Image Visualization (\mref{M_vizResultImg})}
\begin{description}
\item[Secrets:]The display (scaling, colours, styling) and drawing stage 
  of the resulting image.
\item[Services:]Given the imaging parameters (raster grid, spot size and shape -
  as provided by modules \mref{M_params}, \mref{M_vizSpotProfile}, and \mref{M_vizSpotContent})
  and the ground truth image (provided by \mref{M_imgGTInput}),
  the module should display a rendering of the full ground truth image resampled
  using the conditions as in \mref{M_vizResultSub} but extended to the full image area
  instead of the restricted bounded subregion area (meaning a raster grid with the same
  relative cell size, but more rows and columns for the larger area covered).
\item[Implemented By:] \progname{}
\item[Type of Module:] \an{[Record, Library, Abstract Object, or Abstract Data Type]
  [Information to include for leaf modules in the decomposition by secrets tree.]}
\end{description}



\subsection{Software Decision Module}

\begin{description}
\item[Secrets:] The design decision based on mathematical theorems, physical
  facts, or programming considerations. The secrets of this module are
  \emph{not} described in the SRS.
\item[Services:] Includes data structure and algorithms used in the system that
  do not provide direct interaction with the user. 
  % Changes in these modules are more likely to be motivated by a desire to
  % improve performance than by externally imposed changes.
\item[Implemented By:] --
\end{description}

\an{==== module template START ====}
\subsubsection{Example Software Module (\mref{M_Input})}
\begin{description}
\item[Secrets:]The format and structure of the input data.
\item[Services:]Converts the input data into the data structure used by the
  input parameters module.
\item[Implemented By:] [Your Program Name Here]
\item[Type of Module:] \an{[Record, Library, Abstract Object, or Abstract Data Type]
  [Information to include for leaf modules in the decomposition by secrets tree.]}
\end{description}
\an{==== module template END ====}


\section{Traceability Matrix} \label{SecTM}

This section shows two traceability matrices: between the modules and the
requirements and between the modules and the anticipated changes.

% the table should use mref, the requirements should be named, use something
% like fref
\begin{table}[H]
\centering
\begin{tabular}{p{0.2\textwidth} p{0.6\textwidth}}
\toprule
\textbf{Req.} & \textbf{Modules}\\
\midrule
R1 & \mref{M_HdwHide}, \mref{mInput}, \mref{mParams}, \mref{mControl}\\
R2 & \mref{mInput}, \mref{mParams}\\
R3 & \mref{mVerify}\\
R4 & \mref{mOutput}, \mref{mControl}\\
R5 & \mref{mOutput}, \mref{mODEs}, \mref{mControl}, \mref{mSeqDS}, \mref{mSolver}, \mref{mPlot}\\
R6 & \mref{mOutput}, \mref{mODEs}, \mref{mControl}, \mref{mSeqDS}, \mref{mSolver}, \mref{mPlot}\\
R7 & \mref{mOutput}, \mref{mEnergy}, \mref{mControl}, \mref{mSeqDS}, \mref{mPlot}\\
R8 & \mref{mOutput}, \mref{mEnergy}, \mref{mControl}, \mref{mSeqDS}, \mref{mPlot}\\
R9 & \mref{mVerifyOut}\\
R10 & \mref{mOutput}, \mref{mODEs}, \mref{mControl}\\
R11 & \mref{mOutput}, \mref{mODEs}, \mref{mEnergy}, \mref{mControl}\\
\bottomrule
\end{tabular}
\caption{Trace Between Requirements and Modules}
\label{TblRT}
\end{table}

\begin{table}[H]
\centering
\begin{tabular}{p{0.2\textwidth} p{0.6\textwidth}}
\toprule
\textbf{AC} & \textbf{Modules}\\
\midrule
\acref{AC_hardware} & \mref{M_HdwHide}\\
\acref{AC_input} & \mref{mInput}\\
\acref{AC_inFormat} & \mref{mParams}\\
\acref{AC_inVerif} & \mref{mVerify}\\
\acref{AC_outFormat} & \mref{mOutput}\\
\acref{AC_outVerif} & \mref{mVerifyOut}\\
\acref{AC_calc} & \mref{mODEs}\\
\acref{AC_displayUnits} & \mref{mEnergy}\\
\acref{AC_simNoise} & \mref{mControl}\\
\acref{AC_imgMetricAlgos} & \mref{mSeqDS}\\
\acref{AC_compositing} & \mref{mSolver}\\
\acref{AC_calcSignal} & \mref{mPlot}\\
\acref{AC_GUI} & \mref{mPlot}\\
\bottomrule
\end{tabular}
\caption{Trace Between Anticipated Changes and Modules}
\label{TblACT}
\end{table}

\section{Use Hierarchy Between Modules} \label{SecUse}

In this section, the uses hierarchy between modules is
provided. \citet{Parnas1978} said of two programs A and B that A {\em uses} B if
correct execution of B may be necessary for A to complete the task described in
its specification. That is, A {\em uses} B if there exist situations in which
the correct functioning of A depends upon the availability of a correct
implementation of B.  Figure \ref{FigUH} illustrates the use relation between
the modules. It can be seen that the graph is a directed acyclic graph
(DAG). Each level of the hierarchy offers a testable and usable subset of the
system, and modules in the higher level of the hierarchy are essentially simpler
because they use modules from the lower levels.

\begin{figure}[H]
\centering
%\includegraphics[width=0.7\textwidth]{UsesHierarchy.png}
\caption{Use hierarchy among modules}
\label{FigUH}
\end{figure}

%\section*{References}

\bibliographystyle {plainnat}
\bibliography{../../../refs/References,../../../refs/cas741,../../../refs/programming}

\newpage{}

\end{document}