\documentclass{article}

\usepackage{tabularx} \usepackage{booktabs}
\title{Problem Statement and Goals\\\progname}

\author{\authname}

\date{}

%% Comments

\usepackage{color}

\newif\ifcomments\commentstrue %displays comments
%\newif\ifcomments\commentsfalse %so that comments do not display

\ifcomments
\newcommand{\authornote}[3]{\textcolor{#1}{[#3 ---#2]}}
\newcommand{\todo}[1]{\textcolor{red}{[TODO: #1]}}
\else
\newcommand{\authornote}[3]{}
\newcommand{\todo}[1]{}
\fi

\newcommand{\wss}[1]{\authornote{blue}{SS}{#1}} 
\newcommand{\plt}[1]{\authornote{magenta}{TPLT}{#1}} %For explanation of the template
\newcommand{\an}[1]{\authornote{cyan}{Author}{#1}}
 %% Common Parts

\newcommand{\progname}{ImgBeamer} % PUT YOUR PROGRAM NAME HERE
\newcommand{\authname}{Joachim de Fourestier} % AUTHOR NAMES                  

\usepackage{hyperref}
    \hypersetup{colorlinks=true, linkcolor=blue, citecolor=blue, filecolor=blue,
                urlcolor=blue, unicode=false}
    \urlstyle{same}
                                

\begin{document}

\maketitle

\begin{table}[hp] \caption{Revision History} \label{TblRevisionHistory}
\begin{tabularx}{\textwidth}{llX} \toprule \textbf{Date} & \textbf{Developer(s)}& \textbf{Change}\\ \midrule 2023/01/20 & Joachim de Fourestier & First
version\\ %Date2 & Name(s) & Description of changes\\ ... & ... & ...\\
\bottomrule \end{tabularx} \end{table}

\section{Problem Statement}
\subsection{Problem}
In the realm of Electron Microscopy (EM), or even images in general, there is a
common misconception that a sharper image is generally more desirable or better
because it gives the impression of better “resolution” or {\it clarity}.
However, that is not always the case.

The concern is with the perceived image {\it quality} and sampling, not
resolution. Resolution can be defined as the minimum distance by which two
points can be distinguished, known as the Rayleigh criterion
\cite{blackburn_microscopy_2020}. The meaning of the term “image resolution” can
be ambiguous since there are many “types” of image resolutions. The main two
types of concern are spatial resolution and pixel resolution (or pixel count).
Spatial resolution of an image pertains to the actual physical dimensions of an
image where the size of the pixels can be expressed in physical units such as
micrometers or nanometers. Conversely, “image resolution” may be used to simply
express the “image size” (such width and height) in terms of {\it virtual} units
(e.g., number of “pixels”).

Images in an SEM are produced using electron optics where electromagnetism is
used to deflect and focus a very fine and continuous beam of electrons
(produced by an electron gun) within a vacuum chamber over a sample surface. This
electron beam is then {\it raster scanned} over discrete locations on the surface
and interacts with a volume within the sample (called interaction
volume)\cite{goldstein_image_2018}. From this interaction volume, different
types of radiation are generated, such as Backscattered Electrons (BSE) and
Secondary Electrons (SEs). Based on the electron beam’s energy and the sample
nature, the interaction volume may grow larger or smaller depending how much the
electrons scatter. Detectors then collect the electrons and convert them into
{\it signals} based on their energy and count. That said, “extra” electrons (of
generally lower energy) may be produced by cascading interactions as they
scatter around against the sample, the chamberwalls, the sample stage, and other
apparatus leading to {\it noise}.

The resolution of a Scanning Electron Microscope (SEM) can be improved by
aberration corrections and other advanced techniques \cite{joy_dc_2005}.
Besides aberrations and many other factors, the resolution and quality of an SEM
image is inherently limited by the beam spot size, the interaction volume, the 
scanning pattern, and the Signal-to-Noise Ratio (SNR). There is a general inverse
relationship between resolution and SNR. For example, an image with a finer beam
will produce an image of higher resolution but will have less signal resulting in a noisier
image (low SNR). An image with a larger beam diameter will result in more
signal(higher SNR), but this image will be of lower resolution
\cite{jeol_scanning_2013}.

Apart from resolution, the sampling pattern also affects the image quality.
Each discrete location over which the beam scans, the spot profile (size and
shape ofbeam) with respect to the pixel size (discrete location with the raster
grid) may overlap already covered areas of the sample as it travels between each
discrete location. This may result in over-sampling, producing a blurrier
image. If the spot is too small, there is little or no overlap, the sample is
under-sampled and the resulting image may look sharper (or sometimes more
pixelated), but there can be a significant loss of information. Some may want
to under-sample to make the image to make it look sharper, but this may lead to a
loss of information.

The quality and information content of an image is important because the lack of
it may ultimately lead to misinterpretation or flawed research.

\subsection{Inputs and Outputs}
\subsubsection{Inputs}
The inputs will be given or preloaded images (representing sample surface
ground truth), the spot profile (shape and relative size with respect to the
output pixel or cell size), the raster grid parameters (split the image into how
many cells or pixels, i.e., rows and columns), and optional subregion of the
full image.
\subsubsection{Outputs} The outputs will be two generated images
from resampling based the given image and parameters: a subregion preview and
the resampled full image.

\subsection{Stakeholders}
The stakeholder or interested parties may include
students, professors, and researchers that are involved with SEM. This may
include disciplines such as materials science, biology, geology, and
semiconductors.

\subsection{Environment}
Cross platform (i.e., Windows, Linux, or Mac), 
on any modern laptop or desktop computer.

\section{Goals}
\begin{itemize}
  \item Easy-to-use (for anyone with basic EM knowledge) and accessible (no
  installer or initial setup needed) tool to help understand the SEM image
  formation process and the interplay between
  \begin{itemize}
  \item Imaging parameters: spot size, shape, overlap, pixel size, and resolution
  (pixel count).
  \item Information content (type of image, e.g., grayscale, brightness and
  contrast, bit depth)
  \end{itemize}
  \item Show how (relatively) the spot profile (size and shape) vs.\ pixel size
  can affect the resulting image.
  \begin{itemize}
    \item This could be used to simulate defocus and astigmatism.
  \end{itemize}
  \item Better elucidate the issue of oversampling vs. undersampling, similarly
  to the Nyquist sampling rate where:
  \begin{itemize}
    \item Undersampling may lead to missed or last information.
    \item Oversampling may lead to over-averaging or obscuring information.
  \end{itemize}
  \item Different spot profiles
  \begin{itemize}
    \item Intensities
    \begin{itemize}
    \item Uniform / Flat distribution
    \item Normal / Gaussian distribution
    \end{itemize}
    \item Shapes and size
    \begin{itemize}
    \item Circle
    \item Ellipse
    \end{itemize}
  \end{itemize}
\end{itemize}

\section{Stretch Goals}
\begin{itemize}
  \item Concept of scale in physical units (e.g., mm, µm, nm)
  \item Support multiple channels simultaneously, such both an image for BSE
  and SE signals.
  \item Additional spot shapes
  \begin{itemize}
  \item Rings
  \item Halo shaped beam with “bright” middle
  \item Polygons
  \end{itemize}
  \item Simulate simplified physics of sample and beam interactions.
  \begin{itemize}
  \item Simulate basic noise or SNR.
  \item Charging effects
  \item Sample topography and “Edge effect” from SEs
  \item Different classes or nature of samples, e.g., alloys, polymers,
  resin-embedded bacteria, integrated circuits
  \end{itemize}
\end{itemize}

\bibliographystyle {abbrv}
\bibliography{../../refs/SEM_Image_chapter}
\end{document}