\documentclass[12pt, titlepage]{article}

\usepackage{colortbl}
\usepackage{booktabs}
\usepackage{tabularx}
\usepackage{hyperref}
\hypersetup{
    colorlinks,
    citecolor=blue,
    filecolor=black,
    linkcolor=red,
    urlcolor=blue
}
\usepackage[round]{natbib}

%% Comments

\usepackage{color}

\newif\ifcomments\commentstrue %displays comments
%\newif\ifcomments\commentsfalse %so that comments do not display

\ifcomments
\newcommand{\authornote}[3]{\textcolor{#1}{[#3 ---#2]}}
\newcommand{\todo}[1]{\textcolor{red}{[TODO: #1]}}
\else
\newcommand{\authornote}[3]{}
\newcommand{\todo}[1]{}
\fi

\newcommand{\wss}[1]{\authornote{blue}{SS}{#1}} 
\newcommand{\plt}[1]{\authornote{magenta}{TPLT}{#1}} %For explanation of the template
\newcommand{\an}[1]{\authornote{cyan}{Author}{#1}}

%% Common Parts

\newcommand{\progname}{ImgBeamer} % PUT YOUR PROGRAM NAME HERE
\newcommand{\authname}{Joachim de Fourestier} % AUTHOR NAMES                  

\usepackage{hyperref}
    \hypersetup{colorlinks=true, linkcolor=blue, citecolor=blue, filecolor=blue,
                urlcolor=blue, unicode=false}
    \urlstyle{same}
                                


\begin{document}

\title{System Verification and Validation Plan for \progname{}} 
\author{\authname}
\date{\today}
	
\maketitle

\pagenumbering{roman}

\section{Revision History}

\begin{tabularx}{\textwidth}{p{3cm}p{2cm}X}
\toprule {\bf Date} & {\bf Version} & {\bf Notes}\\
\midrule
2023/02/17 & 1.0 & First version \\
Date 2 & 1.1 & Notes\\
\bottomrule
\end{tabularx}

\newpage

\tableofcontents

\listoftables
\wss{Remove this section if it isn't needed}

\listoffigures
\wss{Remove this section if it isn't needed}

\newpage

\section{Symbols, Abbreviations and Acronyms}

\renewcommand{\arraystretch}{1.2}
\begin{tabular}{l l} 
  \toprule		
  \textbf{symbol} & \textbf{description}\\
  \midrule 
  GUI & Graphical User Interface\\
  SRS & Software Requirements Specification\\
  T & Test\\
  VnV & Verification and Validation\\
  \bottomrule
\end{tabular}\\

\wss{symbols, abbreviations or acronyms --- you can simply reference the SRS
\citep{SRS} tables, if appropriate}

\wss{Remove this section if it isn't needed}

\newpage

\pagenumbering{arabic}

This document lays out the Verification and Validation (VnV) plan of the \progname{} (scanning electron 
microscope image formation) as described by the Software Requirements Specification (SRS) 
document \citep{SRS}. Testing of the software and its components is conducted to build confidence in 
its usability, accuracy, and ultimately whether it meets the SRS.

\section{General Information}

\subsection{Summary}

The \progname{} software aims to be a demonstration and learning tool of how a scanning electron 
microscope (SEM) formulates an image. The idea is to help visualize the influence trends of imaging 
parameters (as defined in the SRS) on image quality (or clarity) and resolution.


\subsection{Objectives} \label{sec_objectives}

The objective is to produce a tool that is easy to use (\textit{usability}). It should feel intuitive 
to the user and should provide easy to understand information. This is so that the user may 
identify any trends in how the final image is affected with relative ease and confidence. As a 
secondary goal, the \textit{accuracy} in the trends are important. As such, the image metrics 
should be straight-forward to understand and provide some kind of assurance away from the 
subjective nature of image quality.

\subsection{Relevant Documentation}

There are multiple design documents that provide the important and intimate details to understand 
some of the concepts that are being tested. These documents are the following:

\begin{itemize}
  \item Problem Statement \citep{Prob_Statement}
  \item SRS \citep{SRS}
  \item VnV Report \citep{VnV_report}
\end{itemize}

\section{Plan}

In this section, multiple plans are described to test and inspect the software with an emphasis 
on \textit{usability}. Multiple approaches and perspectives will be employed by the VnV team (\ref{sec_vnv_team})
to help build confidence in the requirements, to avoid any missed important details, 
and to deliver on the qualities as mentioned in the objective (\ref{sec_objectives}).

\subsection{Verification and Validation Team} \label{sec_vnv_team}

The members of the VnV team as well their individual roles are listed in the following table:

\begin{table}[h!]
  \centering
  \begin{tabular}{|r|l|}
    \rowcolor[gray]{0.9}
    \hline
    \textbf{Role} & \textbf{Name} \\ \hline
    Project Supervisor & Dr.\ Spencer Smith  \\ \hline
    Author             & \authname           \\ \hline
    Domain Expert      & Karen Wang          \\ \hline
    SRS Reviewer       & Jason Balaci        \\ \hline
    VnV Plan Reviewer  & Sam Crawford        \\ \hline
    MG + MIS Reviewer  & Lesley Wheat        \\ \hline
    Expert Consultant  & Dr. Nabil Bassim    \\ \hline
    Expert Consultant  & Michael W. Phaneuf  \\ \hline
  \end{tabular}
  \caption{Table of the VnV Team Members}
  \label{table_vnv_team}
\end{table}

\an{Dr. Bassim is my PhD Supervisor, and Mr. Phaneuf is my co-supervisor (employer/industry), 
since I am doing an industrial PhD. Much of this project was originally conceptualized with 
the help and guidance of Mr. Phaneuf.}


\subsection{SRS Verification Plan}

The SRS document is reviewed by the project supervisor, the SRS reviewer and the author. Some input
may be given by the expert consultants if time permits. Most of the feedback has been provided 
as issues on GitHub, or as annotated documents, or by verbal exchange. The author is then expected
make the resolve the issues and makes accordingly throughout the development of the project.
The key objectives are to verify that the software requirements and the documentation are coherent
and sound to the intended audience as defined in the SRS.


\subsection{Design Verification Plan}

Much of the conceptualization was done after having multiple discussions with the expert 
consultants and having done literature review for preexisting tools and documentation 
relevant to the project. Decision are made with focus on the usability, with little to 
no setup being required. The design and implementation is documented in the 
MG / MIS \citep{MG,MIS}.
The VnV team as well as any volunteer users are welcome to provide
their input (through GitHub issues). Eventually, the project may be published in a journal
where the software \progname and its accompanying documentation will 
likely be rigorously reviewed.

\an{Not sure what was appropriate to put here.}


\subsection{Verification and Validation Plan Verification Plan}

The goal is to uncover any mistakes and reveal any risks through the supervision and 
review of the VnV team members. Once most of the work has been done, the work and
accompanying documentation shall undergo a final vetting process. Mainly, the team
will check whether the documented testing plans and verification process have been 
accomplished and the requirements fulfilled.


\subsection{Implementation Verification Plan}

Much of the software will be tested manually by users. This will include checking for
any inconsistencies, bugs in the graphical user interface (GUI), and any unexpected
artifacts in images produced.
The image metrics will be tested using unit tests (section \ref{sec_unittest}). For the
all the code implemented, linters will be used as mentioned in section \ref{sec_autotest_tools}.
As a control for the image metrics, they will be calculated using the same image as the ground truth 
reference to rule out any baseline or unexpected factors in the implementations themselves.


\subsection{Automated Testing and Verification Tools} \label{sec_autotest_tools}

The image quality metric shall be unit-tested using \href{https://pytest.org}{pytest} for 
automated testing of any algorithms implemented in Python and \href{https://qunitjs.com}{QUnit} 
shall be used for those implemented in javascript. The unit tests are listed in 
section \ref{sec_unittest}.
As for linter, \href{https://flake8.pycqa.org}{flake8} shall be used for Python code 
and \href{https://eslint.org}{ESLint} for javascript code.
The \href{https://github.com/andrewekhalel/sewar}{sewar} python package will be 
used as a reference implementation in Python for the image quality metrics.


\subsection{Software Validation Plan}
A \textit{usability} survey will be conducted to evaluate the user experience and whether 
the GUI is intuitive enough to the intended users as described in the SRS \citep{SRS}.
An \textit{accuracy} survey will be conducted to assess the user-perceived image quality. 
The trends identified in the surveys results will be compared to the calculated image metrics.
As a control for the images produced, a manual and an automated test will be conducted to verify if 
an image identical (or with unperceivable difference) to ground truth image can be reproduced.
The compared images shall be in the \textit{accuracy} survey for confirmation. This can be compared
as well using the image metrics. 

\an{I think I might be confusing the Software and the Implementation validation plans...}

\section{System Test Description}
	
\subsection{Tests for Functional Requirements}

\wss{Subsets of the tests may be in related, so this section is divided into
  different areas.  If there are no identifiable subsets for the tests, this
  level of document structure can be removed.}

\wss{Include a blurb here to explain why the subsections below
  cover the requirements.  References to the SRS would be good here.}

\subsubsection{Area of Testing1}

\wss{It would be nice to have a blurb here to explain why the subsections below
  cover the requirements.  References to the SRS would be good here.  If a section
  covers tests for input constraints, you should reference the data constraints
  table in the SRS.}
		
\paragraph{Title for Test}

\begin{enumerate}

\item{test-id1\\}

Control: Manual versus Automatic
					
Initial State: 
					
Input: 
					
Output: \wss{The expected result for the given inputs}

Test Case Derivation: \wss{Justify the expected value given in the Output field}
					
How test will be performed: 
					
\item{test-id2\\}

Control: Manual versus Automatic
					
Initial State: 
					
Input: 
					
Output: \wss{The expected result for the given inputs}

Test Case Derivation: \wss{Justify the expected value given in the Output field}

How test will be performed: 

\end{enumerate}

\subsubsection{Area of Testing2}

...

\subsection{Tests for Nonfunctional Requirements}

\wss{The nonfunctional requirements for accuracy will likely just reference the
  appropriate functional tests from above.  The test cases should mention
  reporting the relative error for these tests.  Not all projects will
  necessarily have nonfunctional requirements related to accuracy}

\wss{Tests related to usability could include conducting a usability test and
  survey.  The survey will be in the Appendix.}

\wss{Static tests, review, inspections, and walkthroughs, will not follow the
format for the tests given below.}

\subsubsection{Area of Testing1}
		
\paragraph{Title for Test}

\begin{enumerate}

\item{test-id1\\}

Type: Functional, Dynamic, Manual, Static etc.
					
Initial State: 
					
Input/Condition: 
					
Output/Result: 
					
How test will be performed: 
					
\item{test-id2\\}

Type: Functional, Dynamic, Manual, Static etc.
					
Initial State: 
					
Input: 
					
Output: 
					
How test will be performed: 

\end{enumerate}

\subsubsection{Area of Testing2}

...

\subsection{Traceability Between Test Cases and Requirements}

\wss{Provide a table that shows which test cases are supporting which
  requirements.}

\section{Unit Test Description} \label{sec_unittest}

\wss{Reference your MIS (detailed design document) and explain your overall
  philosophy for test case selection.}  
\wss{This section should not be filled in until after the MIS (detailed design
  document) has been completed.}

\subsection{Unit Testing Scope}

\wss{What modules are outside of the scope.  If there are modules that are
  developed by someone else, then you would say here if you aren't planning on
  verifying them.  There may also be modules that are part of your software, but
  have a lower priority for verification than others.  If this is the case,
  explain your rationale for the ranking of module importance.}

\subsection{Tests for Functional Requirements}

\wss{Most of the verification will be through automated unit testing.  If
  appropriate specific modules can be verified by a non-testing based
  technique.  That can also be documented in this section.}

\subsubsection{Module 1}

\wss{Include a blurb here to explain why the subsections below cover the module.
  References to the MIS would be good.  You will want tests from a black box
  perspective and from a white box perspective.  Explain to the reader how the
  tests were selected.}

\begin{enumerate}

\item{test-id1\\}

Type: \wss{Functional, Dynamic, Manual, Automatic, Static etc. Most will
  be automatic}
					
Initial State: 
					
Input: 
					
Output: \wss{The expected result for the given inputs}

Test Case Derivation: \wss{Justify the expected value given in the Output field}

How test will be performed: 
					
\item{test-id2\\}

Type: \wss{Functional, Dynamic, Manual, Automatic, Static etc. Most will
  be automatic}
					
Initial State: 
					
Input: 
					
Output: \wss{The expected result for the given inputs}

Test Case Derivation: \wss{Justify the expected value given in the Output field}

How test will be performed: 

\item{...\\}
    
\end{enumerate}

\subsubsection{Module 2}

...

\subsection{Tests for Nonfunctional Requirements}

\wss{If there is a module that needs to be independently assessed for
  performance, those test cases can go here.  In some projects, planning for
  nonfunctional tests of units will not be that relevant.}

\wss{These tests may involve collecting performance data from previously
  mentioned functional tests.}

\subsubsection{Module ?}
		
\begin{enumerate}

\item{test-id1\\}

Type: \wss{Functional, Dynamic, Manual, Automatic, Static etc. Most will
  be automatic}
					
Initial State: 
					
Input/Condition: 
					
Output/Result: 
					
How test will be performed: 
					
\item{test-id2\\}

Type: Functional, Dynamic, Manual, Static etc.
					
Initial State: 
					
Input: 
					
Output: 
					
How test will be performed: 

\end{enumerate}

\subsubsection{Module ?}

...

\subsection{Traceability Between Test Cases and Modules}

\wss{Provide evidence that all of the modules have been considered.}
				
\bibliographystyle{plainnat}
\bibliography{../../refs/References,../../refs/cas741}

\newpage

\section{Appendix}

This is where you can place additional information.

\subsection{Symbolic Parameters}

The definition of the test cases will call for SYMBOLIC\_CONSTANTS.
Their values are defined in this section for easy maintenance.

\subsection{Usability Survey Questions?}

\wss{This is a section that would be appropriate for some projects.}

\newpage{}
\section*{Appendix --- Reflection}

The information in this section will be used to evaluate the team members on the
graduate attribute of Lifelong Learning.  Please answer the following questions:

\newpage{}
\section*{Appendix --- Reflection}

The information in this section will be used to evaluate the team members on the
graduate attribute of Lifelong Learning.  Please answer the following questions:

\begin{enumerate}
  \item What knowledge and skills will the team collectively need to acquire to
  successfully complete the verification and validation of your project?
  Examples of possible knowledge and skills include dynamic testing knowledge,
  static testing knowledge, specific tool usage etc.  You should look to
  identify at least one item for each team member.
  \item For each of the knowledge areas and skills identified in the previous
  question, what are at least two approaches to acquiring the knowledge or
  mastering the skill?  Of the identified approaches, which will each team
  member pursue, and why did they make this choice?
\end{enumerate}

\end{document}