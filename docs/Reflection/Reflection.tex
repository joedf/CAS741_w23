\documentclass{article}

\usepackage{tabularx}
\usepackage{booktabs}

\title{Reflection Report on \progname}

\author{\authname}

\date{\today}

%% Comments

\usepackage{color}

\newif\ifcomments\commentstrue %displays comments
%\newif\ifcomments\commentsfalse %so that comments do not display

\ifcomments
\newcommand{\authornote}[3]{\textcolor{#1}{[#3 ---#2]}}
\newcommand{\todo}[1]{\textcolor{red}{[TODO: #1]}}
\else
\newcommand{\authornote}[3]{}
\newcommand{\todo}[1]{}
\fi

\newcommand{\wss}[1]{\authornote{blue}{SS}{#1}} 
\newcommand{\plt}[1]{\authornote{magenta}{TPLT}{#1}} %For explanation of the template
\newcommand{\an}[1]{\authornote{cyan}{Author}{#1}}

%% Common Parts

\newcommand{\progname}{ImgBeamer} % PUT YOUR PROGRAM NAME HERE
\newcommand{\authname}{Joachim de Fourestier} % AUTHOR NAMES                  

\usepackage{hyperref}
    \hypersetup{colorlinks=true, linkcolor=blue, citecolor=blue, filecolor=blue,
                urlcolor=blue, unicode=false}
    \urlstyle{same}
                                


\begin{document}

\maketitle

\section{Changes in Response to Feedback}

\subsection{SRS}
This course really drove me to better explain and share my thoughts. 
I realize now that there are never too many figures. A good visual explanation
really does say more than words could express. In response to some confusion,
I create many figures that are now included in the SRS with many
visual examples with a concise explanation for each of them.
The wording for the problem and its solution was made a clearer.
I had many minor issues with typos, inconsistencies with acronyms, and
some missing and incomplete definitions.
Following the "mantra" of "design for change",
I realize the importance of making requirements and
assumptions as atomic as possible: not only is it easier to reference, but it
is easier to update.

\subsection{Design and Design Documentation}
By separating the responsibilities of the software into so-called modules,
the code was updated to match and is now much clearer than it was before.
As a result, it was easier for me to understand and see what 
other changes were needed to improve maintainability.
Needed changes (or identified flaws) that were
not completed (or fixed) were at least noted in the
code as ``to-do'' comments. Otherwise, some global variables were not even used
and have been removed.

\subsection{VnV Plan and Report}
Direct feedback helped make a clearer testing plan that I then followed myself.
Had I not taken the time to address the changes to the VnV plan, I likely
would not have found the several inconsistencies and small issues as noted
in the VnV report. I cannot imagine how much worse it could have been otherwise,
and if someone else had done the testing without these changes or the knowledge
I have.
The tests should be numerous and each input and output needs to be justified.
It is important to include tests even if we \textit{know} the ``answer'', since
more often than not we are can be wrong and miss things... which I did ;-)


\section{Design Iteration}
The first version was very rudimentary. It was only after multiple interactions
with the "expert consultants" that the needs became clearer for all parties involved.
This really goes to show what Dr.~Parnas meant why a "rational design" process
needs to be "faked". Some requirements and even assumptions do not become clear
till much later on. The original design was going to be a simple input-output
program with two numbers: a pixel size and a relative spot size. This has now
evolved into an interactive visualization program that allows for different
spot shapes and display each part of the sampling process.

\section{Design Decisions}
The original software was likely going to be in python due to the large
repository of helpful libraries and preloaded modules.
However, the idea of a web application came to mind, because of the
built-in cross-platform design and its incredible
ease to sharing information / webpages.
It is \textit{by design}...
During implementation, instead of just of displaying the resulting image,
it became clear that it was very beneficial
to show each step of the sampling and image formation process.

\section{Reflection on Project Management}
One of the most the difficult tasks for me, is to estimate how much effort and time
some things take. As a software developer, it is always so easy to fall into the 
"trap" of feature creep. A scope needs to be well-defined, realistic, and justified:
"done one thing and do it right". Over-committing and not having a realistic 
scope can ultimately lead to the failure of the whole project. In some sense,
I guess this is about "modular" project planning or risk management,
where smaller manageable tasks mean smaller manageable possible failures.

\subsection{What Went Well?}
By splitting up the design into separate documents with a clear purpose for each,
it became clearer what was priority and what was not. This helped define the
project scope. As for testing, I was relatively surprised what I could find.
Even though most of the issues were minor, their sum effect lead to subtle
but noticeable issues that I likely would have otherwise dismissed as
simple "rounding issues" or "cosmic rays"...

\subsection{What Went Wrong?}
Naturally, I was overly ambitious with the original design with the time allotted.
Much of the testing had to be limited and some features I wanted
to implement had to be put aside, such as a coordinate system
or noise simulation. 

\subsection{What Would you Do Differently Next Time?}
I realize that it is sometimes better to go ahead with small but wrong design decisions
and move forward than to hesitate. Not everything is knowable from the start.
I need to accept that we will to go back and change things.
Next time, I should gradually grow of a list of potential items ordered by priority,
but \textbf{estimate} the time (with extra leeway) it takes for each.
Then, take the pragmatic approach of limiting or
truncating the list to the time allotted.

\end{document}