\documentclass{article}

\usepackage{tabularx}
\usepackage{booktabs}

\title{Reflection Report on \progname}

\author{\authname}

\date{\today}

\input{../Comments}
%% Common Parts

\newcommand{\progname}{ImgBeamer} % PUT YOUR PROGRAM NAME HERE
\newcommand{\prognamelong}{SEM image formation demo tool}
\newcommand{\authname}{Joachim de Fourestier} % AUTHOR NAMES                  

% for code fragments
\newcommand{\code}[1]{\texttt{#1}}

% Useful for tables - top-left centered columns
% https://tex.stackexchange.com/a/337681/289389
\newcolumntype{L}[1]{>{\raggedright\arraybackslash}p{#1}}

\usepackage{hyperref}
    \hypersetup{colorlinks=true, linkcolor=blue, citecolor=blue, filecolor=blue,
                urlcolor=blue, unicode=false}
    \urlstyle{same}
                                


\begin{document}

\maketitle

\section{Changes in Response to Feedback}

\subsection{SRS}
This course really drove me to better explain and share my thoughts. 
I realize now that there are never too many figures. A good visual explanation
really does say more than words could express. In response to some confusion,
I create many figures that are now included in the SRS with many
visual examples with a concise explanation for each of them.
The wording for the problem and its solution was made a clearer.
I had many minor issues with typos, inconsistencies with acronyms, and
some missing and incomplete definitions.
Following the "mantra" of "design for change",
I realize the importance of making requirements and
assumptions as atomic as possible: not only is it easier to reference, but it
is easier to update.

\subsection{Design and Design Documentation}
By separating the responsibilities of the software into so-called modules,
the code was updated to match and is now much clearer than it was before.
As a result, it was easier for me to understand and see what 
other changes were needed to improve maintainability.
Needed changes (or identified flaws) that were
not completed (or fixed) were at least noted in the
code as ``to-do'' comments. Otherwise, some global variables were not even used
and have been removed.

\subsection{VnV Plan and Report}
Direct feedback helped make clearer testing plan that I then followed myself.
Had I not taken the time to address the changes to the VnV plan, I likely
would not have found the several inconsistencies and small issues as noted
in the VnV report.
The tests should be numerous and each input and output needs to be justified.
It is important to include tests even if we \textit{know} the ``answer'', since
more often than not we are can be wrong and miss things... which I did ;-)


\section{Design Iteration}

\plt{Explain how you arrived at your final design and implementation.  How did
the design evolve from the first version to the final version?} 

\section{Design Decisions}

\plt{Reflect and justify your design decisions.  How did limitations,
 assumptions, and constraints influence your decisions?}

\section{Reflection on Project Management}

\plt{This question focuses on processes and tools used for project management.}

\subsection{What Went Well?}

\plt{What went well for your project management in terms of processes and 
technology?}

\subsection{What Went Wrong?}

\plt{What went wrong in terms of processes and technology?}

\subsection{What Would you Do Differently Next Time?}
I realize that it's better the wrong design decision and move forward than
to hesitate. 

\plt{What will you do differently for your next project?}

\end{document}